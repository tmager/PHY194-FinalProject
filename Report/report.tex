\documentclass[twocolumn,10pt]{article}
\usepackage[utf8]{inputenc}
\usepackage[margin=1.5cm]{geometry}
\usepackage{amsmath}
\usepackage{xfrac}
\usepackage{array}
\usepackage{booktabs}
\usepackage{hyperref}
\usepackage{xcolor}
\usepackage[inline,shortlabels]{enumitem}
\usepackage{graphicx}
\usepackage{wrapfig}
\usepackage{microtype}

\hypersetup{ % Turn off link highlighting
  allbordercolors = {white},
  colorlinks,
  allcolors = {black}
}



\begin{document}
\twocolumn[\centering \Huge Sphere Packing \vspace*{14pt}]

\section*{Overview}


\section*{Variations on the Algorithm}

\subsection*{Convergence Criteria and Multiple Move Attempts}



% Talk about move acceptance ratio graph for a low number of points, and how
% the average over a few iterations doesn't appear to be near zero when the
% algorithm terminates.

\subsection*{Adaptive Inflation}

The second variation of the algorithm that we tried was implementing adaptive
inflation rate, where the particles would increase at some fixed rate up until
the minimum distance 

\section*{Results}

For each variation on our algorithm, we performed 50 runs each with 12, 20,
and 50 particles. We tested 1, 2, 5, and 10 move attempts per particle per
iteration; the adaptive inflation method was tested with both 1 and 10 move
attempts per particle per iteration.

\subsection*{Basic Algorithm}



\subsection*{Multiple-Move-Attempt Algorithm}

As we expected, this algorithm tended to produce somewhat denser packs at the
cost of slightly increased run time. However, there was in general no
improvement in the pack as a result of making more attempts beyond two or
three.

This method of making multiple move attempts did have a downside, however. For
larger numbers of particles, inflation was rapid enough that a single particle
could keep the algorithm from stopping by making tiny moves, with the rest of
the particles failing to make any moves, and thus causing the other particles
to inflate such that they overlapped one another. Needless to say this makes
the pack invalid. Looking at the individual run data in \#\#\#\#\#, we see
that the packing ratios for some of the 50-point, 10-attempt packs even go
above one, which is very obviously wrong.

\subsection*{Adaptive Inflation}

As expected, the adaptive inflation algorithm took massively longer than
anything else we tried for comparable numbers of particles --- at least for
multiple-attempt packs. For smaller numbers of particles, it sometimes
converged as quickly (time-wise) as our other methods when run taking only
single move attempts. It also tended to produce lower packing ratios than any
of the other methods. This can probably be attributed to some combination of
the fact that the adaptive inflation algorithm was the only one that actually
enforced non-overlapping spheres specifically (since having the spheres
overlap would inflate the packing ratio) and that it could terminate if two
spheres happened to end up right next to one another by chance, even if they
have plenty of room to move in other directions.


\section*{Conclusions}




\end{document}
