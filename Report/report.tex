\documentclass[twocolumn,10pt]{article}
\usepackage[utf8]{inputenc}
\usepackage[margin=1.5cm]{geometry}
\usepackage{amsmath}
\usepackage{xfrac}
\usepackage{array}
\usepackage{booktabs}
\usepackage{hyperref}
\usepackage{xcolor}
\usepackage[inline,shortlabels]{enumitem}
\usepackage{graphicx}
\usepackage{wrapfig}
\usepackage{microtype}

\hypersetup{ % Turn off link highlighting
  allbordercolors = {white},
  colorlinks,
  allcolors = {black}
}



\begin{document}
\twocolumn[\centering \Huge Sphere Packing \vspace*{14pt}]

\section*{Overview}


\section*{Variations on the Algorithm}

\subsection*{Convergence Criteria}

One possible optimization that we tried was changing how the algorithm decided
when it had converged. The original code concluded that the algorithm had
converged when no moves were accepted in an iteration. This could, we
believed, caused the algorithm to terminate early, particularly for small
numbers of particles, since there is some chance that no moves would be
accepted even if some particles still had possible moves, because the moves
were generated randomly.

% Talk about move acceptance ratio graph for a low number of points, and how
% the average over a few iterations doesn't appear to be near zero when the
% algorithm terminates.

\subsection*{}


\subsection*{}
\end{document}
